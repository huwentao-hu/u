% 指标阈值与改进建议。
在数值算例和不确定性分析中,我们进一步评估策略效果并提出改进方向。
\begin{figure}[htbp]\centering
  \includegraphics[width=0.9\linewidth]{paper/figs/q3_strategy.png}
  \caption{Q3:决策流程图(将来生成)}
\end{figure}
\begin{figure}[htbp]\centering
  \includegraphics[width=0.9\linewidth]{paper/figs/q3_cost.png}
  \caption{Q3:成本比较(将来生成)}
\end{figure}
\begin{figure}[htbp]\centering
  \includegraphics[width=0.9\linewidth]{paper/figs/q3_cost_dist.png}
  \caption{Q3:成本分布(将来生成)}
\end{figure}
\begin{figure}[htbp]\centering
  \includegraphics[width=0.9\linewidth]{paper/figs/q4_revised_strategy.png}
  \caption{Q4:改进策略示意(将来生成)}
\end{figure}
\begin{figure}[htbp]\centering
  \includegraphics[width=0.9\linewidth]{paper/figs/q4_effect.png}
  \caption{Q4:效果对比(将来生成)}
\end{figure}
\begin{figure}[htbp]\centering
  \includegraphics[width=0.9\linewidth]{paper/figs/q4_risk.png}
  \caption{Q4:风险曲线(将来生成)}
\end{figure}
