%! Mode:: "TeX:UTF-8"
%! TEX program = xelatex
\PassOptionsToPackage{quiet}{xeCJK}

% ===============================================
% 论文模板(已将“论文提示词”按位置融入为注释)
% 使用文档类:cumcmthesis + natbib(数字型),其余宏包按模板。
% 所有【提示】均以注释形式出现(以 % 开头),请按需填写正文并删除提示注释。
% ===============================================

% \documentclass{cumcmthesis}            % 【纸质版论文】
\documentclass[withoutpreface,bwprint]{cumcmthesis} % 【电子版论文】去掉封面与编号页

\BeforeBeginEnvironment{tabular}{\zihao{-5}}

%%%%%%%% 参考文献(与模板一致:natbib 数字型)
\usepackage{etoolbox}
\usepackage[numbers,sort&compress]{natbib}  % 文献管理(数字型)
% 编译顺序建议:xelatex → bibtex → xelatex → xelatex

%%%%%%%% 常用宏包
\usepackage[framemethod=TikZ]{mdframed}
\usepackage{url}
\usepackage{tabularx}
\usepackage{subcaption}
\usepackage{booktabs}
\usepackage{geometry}
\usepackage{longtable}
\usepackage{cleveref} % 智能引用,建议统一用 \cref
\usepackage{amsmath,amssymb,bm}
\usepackage{listings}
\usepackage{enumitem} % 若无此包而使用 \setlist 会报错

% 全局设置:取消所有无序列表缩进(如需)
\setlist[itemize]{leftmargin=0pt, itemindent=*, labelsep=0pt}

% 示例表格库(可选):pgfplotstable 直接读取 CSV
\usepackage{pgfplotstable,tabularx,graphicx,float}

% 与模板一致的新列类型(普通表格与 tabularx 皆可使用)
\newcolumntype{C}{>{\centering\arraybackslash}X}
\newcolumntype{R}{>{\raggedleft\arraybackslash}X} % 没有错,不要质疑
\newcolumntype{L}{>{\raggedright\arraybackslash}X}
% \newcolumntype{}{}  % 需要更多列类型可自行添加

%%%%%%%%%%%%%%%%%%%%%%%%%%%%%%%%%%%%%%%%%%%%%%%%%%%%%%%%%%%%%

% ========== 论文题目信息(可按需修改) ==========
\title{(请在此填写论文标题;格式建议:“基于XXX模型/方法/理论的YYY问题研究”)}
\tihao{C}
\baominghao{202500000000}
\schoolname{某某大学}
\membera{队员A}
\memberb{队员B}
\memberc{队员C}
\supervisor{指导老师}
\yearinput{2025}
\monthinput{09}
\dayinput{12}

%%%%%%%%%%%%%%%%%%%%%%%%%%%%%%%%%%%%%%%%%%%%%%%%%%%%%%%%%%%%%
\begin{document}

% =================== 摘要 ===================
\maketitle

\begin{abstract}
% 【提示|摘要(来自论文提示词)】
% - 字数:800–1000 字,“总—分—总”结构;首尾各≤100字。
% - 内容:围绕“问题—方法—结果—优势”展开;必须给出具体数字(如 R^2、RMSE、节省成本\% 等)。
% - 输出要求:摘要正文 + 关键词(3–5个,包含核心模型与研究对象)。
% - 若需英文版摘要,请对应翻译,并保持数字/符号一致。
%
% (此处填写你的摘要正文…)

\keywords{在此填写 3–5 个关键词;覆盖方法与对象}
\end{abstract}

% % 可选:目录
% \tableofcontents
% \newpage

%%%%%%%%%%%%%%%%%%%%%%%%%%%%%%%%%%%%%%%%%%%%%%%%%%%%%%%%%%%%%

\section{问题重述}
% 【提示|问题重述(来自论文提示词)】
% - 背景(≤250字):结合行业现状/痛点与近3–5年文献(引用1–2篇)。
% - 对每个小问给出 130–150 字的准确描述。
% - 输出:引用处用 \cite{};使用thebibliography ,要与 GB/T 7714-2015 数字型匹配)。
%
% (此处填写“问题重述”的正文…)

\section{问题分析}
% 【提示|问题分析(来自论文提示词)】
% - 采用“总-分”结构:先总述“数据驱动建模 + 多目标优化”的整体框架;
% - 逐小问说明关键矛盾、解题思路与核心步骤(每问 150–200 字)。
% - 结构建议:\subsection{总体分析} 与 \subsection{问题X的分析};每个小问以“\textbf{针对问题X:} …”开头。
% - 可在文末追加流程图。
%
% (此处填写“问题分析”的正文…)

\section{模型假设}
% 【提示|模型假设(来自论文提示词)】
% - 列出 5–7 条“可检验/可放松”的假设;给出带数字列表的 LaTeX 版本。
% - 每条假设建议说明“合理性/潜在影响/如何在敏感性中检验”。
%
% 示例(可删除):
% \begin{enumerate}
%   \item 假设同批次独立同分布,坏率 \theta 服从 Beta 先验;
%   \item 观测噪声在小范围内稳定,可并入误差项;
%   \item 单位成本参数在一个滚动周期内近似常数;
% \end{enumerate}

\section{符号说明}
% 【提示|符号说明(来自论文提示词】
% - 使用 tabularx + booktabs 给出“三线表”,列类型与模板一致(如 L/C/R)。
% 示例代码(按需取消注释并修改):
% \begin{table}[H]
%     \centering
%     \begin{tabularx}{\textwidth}{>{\hsize=0.5\hsize}X >{\hsize=1.8\hsize}X >{\hsize=0.7\hsize}R}
%         \toprule
%         符号 & 含义 & 单位\\
%         \midrule
%         $\theta$ & 单件零件的坏率(不合格概率) & ---\\
%         $\theta^*$ & 目标坏率阈值/接收质量上限(AQL) & ---\\
%         \bottomrule
%     \end{tabularx}
%     \caption{符号说明}
%     \label{tab:符号说明}
% \end{table}

\section{问题一模型建立与求解}
% 【提示|问题一模型建立与求解(来自论文提示词)】
% 【模型建立】
% - 原理概述(灰色/回归/图模型/优化等):用通俗语言解释;
% - 建模步骤:数据预处理 → 参数估计 → 模型构建;
% - 公式(equation/align):目标函数、约束、关键推导(每式给出物理/工程含义)。
%
% 【求解方法】
% - 工具:MATLAB/Python;初始化 → 迭代 → 收敛判定;
% - 结果呈现:最优解/拟合优度/误差;可以插入流程图以进行内容演示;可以插入 2–4 幅中文图表,并在图下写出“关键结论”。
%
% 【结果分析】可以插入 2–5 幅中文图表,并在图下写出“关键结论”。
% - 基础:趋势/波动范围/拟合度;
% - 深层:敏感性/关键因子权重/约束匹配度;
% - 回答题面:指出最优方案或决策阈值。
%
% 【模型检验】可以插入 2–4 幅中文图表,并在图下写出“关键结论”。
% - 方法:残差/显著性/后验差比/K 折交叉验证;
% - 结论:是否达到一级精度或阈值内。
%
% (此处填写问题一的正文…)

\section{问题二模型建立与求解}
% 【提示|问题二:请复用“模型建立—求解—结果—检验”结构,并保证与问题一相同的可复现性与图表规范】
% (此处填写问题二的正文…)

\section{问题三模型建立与求解}
% 【提示|问题三:请复用“模型建立—求解—结果—检验”结构,并保证与问题一相同的可复现性与图表规范】
% (此处填写问题三的正文…)

\section{问题四模型建立与求解}
% 【提示|问题四:请复用“模型建立—求解—结果—检验”结构,并保证与问题一相同的可复现性与图表规范】
% (此处填写问题四的正文…)

\section{模型评价、改进与推广}
% 【提示|模型评价(来自论文提示词)】
% - 优点(3条):创新性/适用性/效率,每条尽量含量化对比(如效率 +40\%)。
% - 缺点(2–3条):数据依赖/线性假设/维度提升慢等。
% - 改进(2–3条):更多数据、非线性项、集成学习、约束增强。
% - 推广(2–3条):相近场景的迁移与参数复用。
%
% (此处填写“模型评价、改进与推广”的正文…)


\section*{参考文献}
% 【提示|参考文献】
% (我这里选择兜底方式B,因为我还要手动添加AI模型引用)

% - 兜底方式 B:手写 thebibliography
% \begin{thebibliography}{7}
% \bibitem{zhang2020bayes} 张伟, 李明. ……
% \end{thebibliography}

% - 推荐方式 A:使用 .bib,命令如下:
%   \bibliographystyle{gbt7714-numerical}
%   \bibliography{refs}  % refs.bib



%%%%%%%%%%%%%%%%%%%%%%%%%%%%%%%%%%%%%%%%%%%%%%%%%%%%%%%%%%%%%
% =================== 附录 ===================
\appendix

\section{支撑材料文件列表}
% 【提示|附录(来自论文提示词)】
% - 提供“文件清单表 + 代码插入”模板;建议统一使用 \lstinputlisting,并在 caption 中写明功能。

% 示例文件清单表(按需取消注释并修改):
% \begin{table}[H]\centering
% \caption{支撑材料文件列表}\label{tab:filelist}
% \begin{tabularx}{\textwidth}{lL}
% \toprule
% 文件名 & 功能描述 \\
% \midrule
% w1.py & 问题一:蒙特卡洛与代价函数 \\
% w2.py & 问题二:POMDP + TS 学习曲线 \\
% \bottomrule
% \end{tabularx}
% \end{table}

\section{代码}
% 【提示|代码清单(来自论文提示词)】
% - 使用 listings:统一 caption/label;相对路径建议放入 code/ 目录;
% - 每段代码须与正文一致(图表、参数、结论可复现);
%
% 示例 \lstinputlisting(按需取消注释并修改):
% \subsection{Python 代码}
% \lstinputlisting[language=Python,caption={问题一脚本示例},label={code:w1}]{code/w1.py}
% \lstinputlisting[language=Python,caption={问题二脚本示例},label={code:w2}]{code/w2.py}

\end{document}